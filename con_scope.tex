\subsubsection{Background}
According to the \gls{ISECG}, the human-robot exploration of Mars is the next ultimate step in space-exploration \cite{RFP}. As a transition phase to Mars, Moon revisiting and asteroids exploration missions are planned by various nations. For example the \gls{ESA} is interested in conducting sample return missions on Phobos or the Moon. The \gls{USA} is currently developing the Orion \gls{MPCV} and the new \gls{SLS} that could serve future human-robotic Mars missions and \gls{ARM} \cite{RFP}.

To facilitate a common platform for the future missions and objectives of different nations, a lunar Outpost stationed at \gls{EML2} was recently proposed. \gls{EML2} is located at the far side of the Moon, an Outpost staged there would provide the direct control of the robotic systems, which perform explorations or constructions on far side of the lunar surface \cite{moon_farside}. The Outpost could also provide service for reusable robotic and human lander systems, initial analysis and curation of lunar sample materials collected from the surface of the moon etc. The Outpost could also serve as a staging area for Mars exploration vehicles assembly because it would reduce the size and number of stages that have to be assembled in orbit \cite{deep_space}.

A robotic system on the Outpost would provide great assistance for the vehicle assembly tasks. More importantly, the robotic system would serve an important role for the construction and maintenance of Outpost as proven in the International Space Station missions \cite{deep_auto}. Therefore, Canada's next generation robot is considered to provide service and assistance on the Outpost.

\subsubsection{Goals and Objectives}
On the Outpost, Canada's next generation robotic system will play an important role, facilitating the functions of crew members and space vehicles. The key qualitative goals of this system include versatility, adaptability, advanced mobility and dexterity, as well as the ability to be upgradeable.\cite{RFP} A multi-purpose system like this will assist Canada and other nations to expand and develop their space exploration programs. 

The design of a space robotic system will revolve around objectives of the mission. The first objective includes supporting tasks involving the following five key operations \cite{RFP} :
\begin{enumerate}
\item{Outpost Reconfiguration}
\item{Outpost Inspection, Maintenance and Repair}
\item{Capture and Berthing of visiting vehicles}
\item{EVA transport}
\item{Contingency Operations}
\end{enumerate}

Another objective is staying within the constraints defined in the \gls{RFP} (further discussed in \Cref{sect:policies}), with a particular focus on the maximum mass and volume. The third objective would be to meet the quantitative requirements laid out in the \gls{RFP} \cite{RFP}.